%% ---------------------------------------------------------------------------
%% intro.tex
%%
%% Introduction
%%
%% $Id: intro.tex 1477 2010-07-28 21:34:43Z palvarado $
%% ---------------------------------------------------------------------------

\chapter{Introducción}
\label{chp:intro}

Coloque dos párrafos de contexto de proyecto

\section{Descripción general del proyecto}
Realice una breve reseña sobre qué se tratará su proyecto

\section{Antecedentes}



\subsection{Descripción de la empresa}

\subsection{Áreas de conocimiento}
Indique en cuales áreas de conocimiento de computadores estará el proyecto. Debe indicar cómo se relaciona de manera específica.

\subsection{Trabajos similares}

Indique al menos tres trabajos similares con sus respectivas referencias, debe de indicar cómo se relacionan y cómo se diferencian de cada uno.

\section{Planteamiento del problema}



\subsection{Contexto del problema}

 

\subsection{Justificación del problema}



\subsection{Definición concreta del problema}
Un único párrafo


\section{Objetivos del proyecto}
Recordar que todos los objetivos deben contener el qué, el para y el cómo


\subsection{Objetivo general}
El objetivo general responde de manera directa el problema en concreto

\subsection{Objetivos específicos}

\begin{enumerate}
    \item 
    \item 
    \item 
\end{enumerate}

\section{Alcances, entregables y limitaciones del proyecto.}


\subsection{Alcances}


\subsection{Entregables}
Realice una tabla en donde se evidencie cada entregable para cada objetivo


\subsection{Limitaciones del proyecto}


%\index{objetivos}

%%% Local Variables: 
%%% mode: latex
%%% TeX-master: "main"
%%% End: 
