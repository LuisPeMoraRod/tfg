%% ---------------------------------------------------------------------------
%% paNotation.tex
%%
%% Notation
%%
%% $Id: paNotation.tex,v 1.15 2004/03/30 05:55:59 alvarado Exp $
%% ---------------------------------------------------------------------------

%\cleardoublepage
\renewcommand{\nomname}{Lista de símbolos y abreviaciones}
\markboth{\nomname}{\nomname}
\renewcommand{\nompreamble}{\addcontentsline{toc}{chapter}{\nomname}%
\setlength{\nomitemsep}{-\parsep}
\setlength{\itemsep}{10ex}
}

\syma{SBD}{Sistema de Banca para el Desarrollo}
\syma{SaaS}{Software como Servicio}
\syma{AUGE}{Agencia Universitaria para la Gestión Emprendedora}
\syma{PIT}{Proyectos de Innovación Tecnológica}
\syma{MVP}{Producto Mínimo Viable}
\syma{CAMTIC}{Cámara de Tecnologías de Información y Comunicación}
\syma{CNN}{Redes Neuronales Convolucionales}
\syma{IoT}{Internet de las Cosas}
\syma{OE}{Objetivo Específico}
\syma{ITCR}{Instituto Tecnológico de Costa Rica}
\syma{API}{Interfaz de Programación de Aplicaciones}
\syma{FPS}{Cuadros por Segundo}
\syma{MMU}{Unidad de Manejo de Memoria}
\syma{BSP}{Paquete de Soporte de Placa}
\syma{LTS}{Soporte a Largo Plazo}
\syma{SDK}{Kit de Desarrollo de Software}
\syma{IaaS}{Infraestructura como Servicio}
\syma{PaaS}{Plataforma como Servicio}
\syma{BaaS}{Backend como Servicio}
\syma{HOG}{Histograma de Gradientes Orientados}
\syma{FLOPs}{Operaciones de Punto Flotante}
\syma{CDN}{Red de Distribución de Contenidos}
\syma{UAT}{Pruebas de Aceptación de Usuario}

\printnomenclature[20mm]

%%% Local Variables:
%%% mode: latex
%%% TeX-master: "paMain"
%%% End: