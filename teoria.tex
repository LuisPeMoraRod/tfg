\chapter{Marco  de referencia teórico}
\label{ch:marco}

El éxito en el diseño y la implementación de un sistema de control de acceso basado en reconocimiento facial depende, en gran medida, de la solidez conceptual que sustente cada decisión de ingeniería. Por ello, en este capítulo se presentan los conceptos fundamentales que guiarán el desarrollo del sistema. Se articulan las definiciones, los modelos y buenas prácticas necesarias para justificar la arquitectura propuesta, orientar la selección de tecnologías y sustentar los criterios de validación.

En consecuencia con la naturaleza multidisciplinaria del sistema, que combina \textit{hardware} especializado, algoritmos de inteligencia artificial y servicios distribuidos, el capítulo integra conceptos clave de diversas áreas, incluyendo:

\begin{enumerate}
    \item Sistemas embebidos
    \item Sistemas operativos embebidos y \textit{Yocto Project}
    \item Internet de las Cosas, computación en la nube y procesamiento de datos biométricos
    \item Algoritmos de reconocimiento y detección facial
\end{enumerate}

Las secciones posteriores profundizarán en sus respectivos ejes temáticos y se describirán los elementos críticos que orientan la solución propuesta. Además, se analizan las relaciones conceptuales entre los diferentes temas mediante un mapa conceptual. De esta forma, el marco teórico no solo contextualiza el trabajo, sino que también sirve de referencia para contrastar la implementación del proyecto con el estado del arte.

\section{Sistemas embebidos}
Un sistema embebido, como Wolf define en \cite{wolf_embedded_2012}, ``\textit{es un dispositivo que incorpora una computadora programable, pero que no está destinado a funcionar como una computadora de propósito general}''. Es un conjunto de componentes de hardware y software, y quizás componentes mecánicos, dedicados a realizar una función específica, o un conjunto acotado de funciones, dentro de un sistema mayor, y que generalmente opera con recursos limitados y bajo requisitos estrictos de confiabilidad y tiempo real \cite{barr_embedded_1999}.

Los sistemas embebidos son ubicuos en la vida cotidiana, y se encuentran en una amplia variedad de aplicaciones. Su diseño y aplicación son una necesidad fundamental para múltiples áreas de la ingeniería, pues dispositivos como automóviles, teléfonos móviles y electrodomésticos dependen en gran medida de microprocesadores embebidos. Para integrar estos componentes en un sistema, los departamentos de ingeniería deben ser capaces de identificar tareas computacionales específicas dentro del producto, diseñar una plataforma de hardware con capacidad de entrada/salida que permita ejecutar dichas tareas, e implementar el software que las coordine de forma eficiente \cite{wolf_embedded_2012}. 

El diseño de sistemas embebidos enfrenta desafíos significativos que van más allá de la mera funcionalidad. Según Jebril y Abu Al-Haija, los sistemas embebidos deben operar bajo restricciones físicas y temporales estrictas, lo que complica su diseño y verificación \cite{jebril_challenges_2017}. Entre los retos más relevantes destacan: la selección adecuada del hardware para cumplir simultáneamente con los requisitos de desempeño y las limitaciones presupuestarias; el cumplimiento estricto de plazos de respuesta en aplicaciones sensibles al tiempo; la minimización del consumo energético, especialmente en dispositivos alimentados por batería; y la necesidad de asegurar confiabilidad en entornos de operación continua o críticos \cite{wolf_embedded_2012}.  

Adicionalmente, se espera que muchos de estos sistemas sean escalables y actualizables, lo cual añade complejidad al diseño inicial. Estos retos, que no suelen presentarse en la misma magnitud en sistemas de propósito general, requieren un enfoque metodológico riguroso y una comprensión profunda de la interacción entre hardware, software y entorno de operación.

\subsection{Características principales}

Los rasgos distintivos de los sistemas embebidos comúnmente citados en la literatura consultada, se resumen en la siguiente tabla: 

\begin{longtable}{l|p{10cm}}
    \caption{Principales características de los sistemas embebidos} \label{tab:embedded_characteristics} \\
    \hline
    \textbf{Característica} & \textbf{Descripción} \\
    \hline
    \endfirsthead
    
    \multicolumn{2}{c}{\tablename\ \thetable{}: Principales características de los sistemas embebidos (continuación)} \\
    \hline
    \textbf{Característica} & \textbf{Descripción} \\
    \hline
    \endhead
    
    \hline
    \multicolumn{2}{r}{\textit{Continúa en la siguiente página}} \\
    \endfoot
    
    \hline
    \endlastfoot
    
    Restricción de recursos & CPU, memoria y energía limitadas obligan a optimizar código y hardware \cite{henriksson_2006}.\\
    Propósito  específico & El software se diseña para una tarea concreta (p. ej., validar credenciales biométricas) \cite{wolf_embedded_2012}.\\
    Operación en tiempo real & Se exige cumplir plazos máximos de respuesta (determinismo) \cite{shyamasundar_validating_2001}.\\
    Alta confiabilidad & El dispositivo suele operar de forma autónoma y continua, por lo que los fallos son inaceptables \cite{windriver_embedded_security}.\\
    Integración hardware-software & El diseño busca optimizar electrónica, sistema operativo y aplicación de manera conjunta \cite{wolf_embedded_2012}.\\
\end{longtable}


\subsection{Clasificación de los sistemas embebidos según su nivel de complejidad}

Los sistemas embebidos pueden clasificarse según diversos criterios, siendo uno de los más relevantes el nivel de complejidad, el cual abarca tanto aspectos del hardware como del software, así como la interacción con el entorno físico. Respecto a esto, Lee y Seshia definen a los sistemas embebidos como ``\textit{computadoras menos visibles que interactúan con procesos físicos, típicamente en la forma de sistemas ciberfísicos con lazos de retroalimentación entre los procesos físicos y el software}'' \cite{lee_introduction_2017}. Esta interacción impone restricciones y retos específicos de diseño que varían según el tipo de sistema.

Según Wolf \cite{wolf_embedded_2012}, los sistemas embebidos pueden clasificarse en cuatro niveles de complejidad creciente: 
\begin{enumerate} 
    \item \textbf{Sistemas embebidos autónomos simples}, tales como relojes digitales o termostatos, los cuales ejecutan una lógica de control elemental y suelen operar sin conectividad o interacción compleja.
    \item \textbf{Sistemas de tiempo real}, los cuales requieren respuestas dentro de márgenes temporales estrictos. Estos sistemas son típicos en automóviles, dispositivos médicos y control industrial.
    \item \textbf{Sistemas embebidos con múltiples procesadores o coprocesadores}, que integran arquitecturas heterogéneas optimizadas para tareas específicas, como decodificadores multimedia o estaciones base de comunicación.
    \item \textbf{Sistemas embebidos distribuidos}, que incluyen componentes interconectados a través de redes, con alta complejidad de coordinación y sincronización, como los sistemas automotrices modernos y las infraestructuras inteligentes.
\end{enumerate}

Esta categorización puede también relacionarse con el grado de criticidad del sistema. Tal como señalan Lee y Seshia, ``\textit{los sistemas ciberfísicos enfrentan desafíos únicos, como la necesidad de garantizar propiedades temporales en un entorno concurrente, donde la corrección depende del comportamiento tanto computacional como físico}'' \cite{lee_introduction_2017}. Por tanto, mientras los sistemas simples pueden tolerar ciertos errores o retrasos, los sistemas complejos, como los sistemas embebidos distribuidos en aeronáutica o cirugía robótica, requieren garantías formales sobre su comportamiento.

En resumen, el nivel de complejidad de un sistema embebido se encuentra determinado por factores como la criticidad temporal, la concurrencia, la heterogeneidad de hardware y la capacidad de integración en redes de sensores y actuadores. Esta clasificación no solo guía las decisiones de diseño, sino que también establece el enfoque metodológico a seguir en su desarrollo.

\subsection{Aplicación de los sistemas embebidos en control de acceso}
Los sistemas embebidos juegan un papel fundamental en los sistemas modernos de control de acceso, particularmente en aplicaciones donde se requiere una validación rápida, confiable y autónoma de la identidad del usuario. El uso de propósito específico, el bajo consumo energético y la capacidad de operar en tiempo real hacen que los sistemas embebidos sean ideales para este tipo de aplicaciones, donde se debe responder a eventos del entorno físico en un tiempo limitado.

En el estudio de Guerbaoui et al., se presenta una plataforma funcional basada en una Raspberry Pi 3 y una cámara Intel RealSense D415, utilizando clasificadores en cascada Haar para la detección y reconocimiento facial. Este sistema fue desarrollado específicamente para operar en tiempo real y en condiciones variables de iluminación y ángulos de visión, mostrando un desempeño adecuado para aplicaciones de vigilancia en instalaciones sensibles \cite{guerbaoui_2025}.

De manera complementaria, Kalkar et al. proponen una arquitectura distribuida en la que un sistema embebido realiza la detección de rostros a través de cámaras IP, y delega el reconocimiento facial a un servidor computacional. Se emplean modelos de detección como \textit{YOLO} y redes neuronales ligeras (\textit{MobileFaceNets}), lo que permite mantener un equilibrio entre velocidad y precisión. El diseño está orientado a aplicaciones en zonas restringidas como oficinas o centros educativos, donde se requiere monitoreo continuo con capacidad de respuesta en tiempo real \cite{kalkar_2020}.

Finalmente, Hammami y Alhammami desarrollan un sistema de control de acceso robusto que combina reconocimiento facial con códigos QR cifrados, ejecutado completamente sobre una Raspberry Pi 4. Esta solución, además de mejorar la precisión del reconocimiento, permite validar la identidad de visitantes mediante códigos de un solo uso y registra todos los eventos de acceso en una base de datos local, logrando un rendimiento de 8.27 FPS (cuadros por segundo) en condiciones reales de uso \cite{hammami_2024}.

Estos estudios evidencian que los sistemas embebidos son una opción viable y efectiva para implementar soluciones de control de acceso basadas en reconocimiento facial. La combinación de hardware especializado, algoritmos optimizados y la capacidad de operar en tiempo real permiten desarrollar sistemas que cumplen con los requisitos de seguridad y eficiencia necesarios en entornos críticos.

\section{Sistemas operativos embebidos y \textit{Yocto Project}}


Un sistema operativo embebido es un software especializado que proporciona servicios básicos como gestión de hardware, planificación de tareas y comunicación entre procesos, adaptados a las restricciones y necesidades particulares de un sistema embebido. Según Holt y Huang, estos sistemas están diseñados para ofrecer un conjunto mínimo pero suficiente de funciones orientadas a una aplicación específica, con alta eficiencia, bajo consumo de recursos y fiabilidad en tiempo real, permitiendo controlar periféricos y tareas críticas de forma predecible \cite{holt_2018_embedded_os}.

Wang complementa esta definición al señalar que un sistema operativo embebido no es un sistema monolítico, sino un conjunto modular de funciones esenciales que permiten la ejecución coordinada y determinista de múltiples tareas en ambientes con limitaciones físicas y temporales \cite{wang_2023_embedded_RTOS}. Estas características lo diferencian de los sistemas operativos de propósito general, cuya arquitectura y funcionalidad están pensadas para entornos flexibles y de uso amplio, como computadoras personales o servidores.

Con el aumento del poder de cómputo en las plataformas embebidas, Linux se ha convertido en una opción popular para este tipo de aplicaciones, gracias a su flexibilidad, código abierto y ecosistema maduro. Sin embargo, su uso requiere de adaptaciones importantes. Existen variantes como \textit{$\mu$Linux}, pensada para sistemas sin unidad de manejo de memoria (\textit{MMU}), y \textit{RTLinux}, que introduce una capa de emulación para manejar tareas críticas en tiempo real sobre el kernel de Linux estándar \cite{holt_2018_embedded_os} \cite{wang_2023_embedded_RTOS}.

Estas adaptaciones, aunque poderosas, implican un nivel de complejidad elevado en la construcción de un sistema embebido, pues es necesario personalizar diversos aspectos críticos del sistema operativo, como el gestor de arranque, la configuración del kernel, la gestión de memoria y la integración de controladores específicos para el hardware utilizado. En este contexto, herramientas como \textit{Yocto Project} cobran una especial relevancia. 

\subsection{Yocto Project: conceptos básicos y herramientas principales}

El \textit{Yocto Project} es un marco de trabajo de código abierto que permite crear distribuciones GNU/Linux adaptadas a sistemas embebidos mediante la generación reproducible de imágenes y paquetes binarios \cite{yocto_overview_2025}. Su piedra angular es el \textbf{OpenEmbedded Build System}, cuyo motor de ejecución es \textit{BitBake}.  A grandes rasgos, los elementos fundamentales que es necesario comprender son los siguientes:

\begin{itemize}
    \item \textbf{BitBake}: herramienta de orquestación que interpreta \emph{meta‐datos} y ejecuta un grafo de tareas (parsing, configuración, compilación, empaquetado, etc.) respetando dependencias y permitiendo compilación en paralelo \cite{bitbake_manual_2025}.  Conceptualmente, es similar a la herramienta \texttt{make} en \texttt{C}, pero está especializada en la construcción de stacks embebidos y en la gestión de cruces de arquitectura (\emph{cross‑compilation}).
      
    \item \textbf{Poky}: distribución de referencia que incluye BitBake, un conjunto mínimo de capas (\texttt{meta}, \texttt{meta‐poky}, \texttt{meta‐yocto‐bsp}, entre otras) y herramientas auxiliares.  Poky sirve de punto de partida para crear distribuciones personalizadas evitando partir de cero \cite{yocto_overview_2025}. Esta herramienta es especialmente útil para desarrolladores que buscan una solución rápida y funcional, ya que proporciona una base sólida sobre la cual construir.
    
    \item \textbf{Capas (\emph{layers})}: estructura modular donde se agrupan colecciones coherentes de meta‑datos.  Cada capa encapsula una política (p.\,ej.\ \textit{Board Support Package}\footnote{Un \textit{Board Support Package (BSP)} es un conjunto de controladores, bibliotecas y firmware específicos que permiten que un sistema operativo funcione en una plataforma de hardware particular. En el contexto de Yocto, un BSP incluye los componentes necesarios para adaptar Linux a un hardware específico como Raspberry Pi o BeagleBone.}, distro, aplicación) y puede activarse o desactivarse de forma independiente, lo que favorece la escalabilidad y el mantenimiento.  El modelo de capas fomenta la separación entre hardware (BSP), políticas de distribución y aplicaciones \cite{yocto_overview_2025}.
    
    \item \textbf{Meta‑datos y \emph{recipes}}: los ficheros \verb|.bb| (\emph{recipes}) describen cómo obtener el código fuente, aplicar parches, compilar y empaquetar un componente.  Los ficheros \verb|.bbclass| (clases) encapsulan lógica común reutilizable, mientras que los \verb|.conf| establecen configuración global (máquina, distro, usuario).  Todo este conjunto constituye los \textit{meta-datos} que BitBake analiza para generar el grafo de tareas \cite{bitbake_manual_2025}.
    
    \item \textbf{Variables y sobrecargas}: la flexibilidad del sistema radica en la extensiva parametrización mediante variables (p.\,ej.\ \texttt{SRC\_URI}, \texttt{DEPENDS}) y en los mecanismos de sobrecarga (\texttt{override}) que permiten adaptar el comportamiento según máquina, arquitectura o política de distribución \cite{bitbake_manual_2025}.  
    \item \textbf{SDK y eSDK}: el flujo de trabajo típico culmina en una imagen para el dispositivo y en un SDK extensible que facilita el desarrollo de aplicaciones en el host, manteniendo coherencia con la cadena de compilación empleada durante el build \cite{yocto_overview_2025}.  
  \end{itemize}

El uso de \textit{Yocto Project} permite a los desarrolladores crear distribuciones personalizadas de Linux para sistemas embebidos, optimizando el proceso de construcción y facilitando la integración de controladores y aplicaciones específicas. Esta flexibilidad es esencial para abordar los desafíos que presentan los sistemas embebidos, donde cada proyecto puede requerir un enfoque único en términos de hardware y software.

\subsection{Ventajas y desventajas de utilizar \textit{Yocto} para sistemas de propósito específico}
El uso de esta herramienta presenta beneficios, pero también desventajas que deben tomarse en cuenta para determinar su idoneidad en un proyecto específico. En la \textit{Tabla \ref{tab:yocto_pros_cons}} se contrastan los puntos más importantes encontrados en la literatura consultada \cite{streif_2016}, \cite{abbott_2018}, \cite{pera_2022}, \cite{karacali_2024}.


\begin{longtable}{p{7.5cm}|p{7.5cm}}
    \caption{Ventajas y desventajas del uso de \textit{Yocto} en sistemas embebidos} \label{tab:yocto_pros_cons} \\
    \hline
    \textbf{Ventajas} & \textbf{Desventajas} \\
    \hline
    \endfirsthead
    
    \multicolumn{2}{c}{\tablename\ \thetable{}: Ventajas y desventajas del uso de \textit{Yocto} en sistemas embebidos (continuación)} \\
    \hline
    \textbf{Ventajas} & \textbf{Desventajas} \\
    \hline
    \endhead
    
    \hline
    \multicolumn{2}{r}{\textit{Continúa en la siguiente página}} \\
    \endfoot
    
    \hline
    \endlastfoot
    
    Flexibilidad y personalización granular mediante el concepto de \textit{layers} y \textit{recipes}, que permite generar distribuciones altamente adaptadas al hardware y a los requisitos de la aplicación & Curva de aprendizaje pronunciada: la complejidad de los metadatos y la multitud de variables/sobrecargas de BitBake exige experiencia avanzada \\
    \hline
    
    Reproducibilidad garantizada: capacidad de recrear exactamente la misma imagen en diferentes momentos y entornos & Tiempo de construcción elevado: incluso con potentes sistemas de desarrollo, la construcción completa puede llevar horas \\
    \hline
    
    Control total del tamaño y características del sistema operativo, eliminando componentes innecesarios y optimizando para entornos con recursos limitados & Requisitos de almacenamiento significativos para el entorno de desarrollo (mínimo 50GB recomendados) \\
    \hline
    
    Soporte extenso para múltiples arquitecturas de hardware (ARM, MIPS, x86, PowerPC, etc.) & Menor disposición de documentación actualizada y casos de uso específicos en comparación con distribuciones genéricas \\
    \hline
    
    Gestión centralizada de dependencias y parches a través del sistema de metadatos & Resolución de problemas compleja debido a la abstracción de las capas y la interacción de múltiples componentes \\
    \hline
    
    Integración con herramientas de seguridad para análisis de vulnerabilidades en imágenes generadas & Mayor esfuerzo inicial de configuración comparado con distribuciones listas para usar como Raspbian o Debian \\
\end{longtable}